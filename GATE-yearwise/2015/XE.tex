\iffalse
\author{EE24BTECH11047}
\section{xe}
\chapter{2015}
\fi
%1
\item The principal presented the chief guest with a \underline{\hspace{1cm}}, as a token of appreciation.
\begin{enumerate}
    \item momento
    \item memento
    \item momentum
    \item moment
\end{enumerate}
\item Frogs \underline{\hspace{1cm}}.
\begin{enumerate}
    \item croak
    \item roar
    \item hiss
    \item patter
\end{enumerate}
\item Choose the word most similar in meaning to the given word:\\
Educe
\begin{enumerate}
    \item Exert
    \item Educate
    \item Extract
    \item Extend
\end{enumerate}
\item Operators $\square, \Diamond$ and $\rightarrow$ are defined by: $a\square b=\frac{a-b}{a+b};a\Diamond b=\frac{a+b}{a-b};a\rightarrow b=ab$. Find the value of $(66\square 6)\rightarrow(66\Diamond 6)$.
\begin{enumerate}
    \item -2
    \item -1
    \item 1
    \item 2
\end{enumerate}
\item If $log_{x}(5/7)=-1/3$, then the value of x is
\begin{enumerate}
    \item 343/125
    \item 125/343
    \item -25/49
    \item -49/25
\end{enumerate}
\item The following question presents a sentence, part of which is underlined. Beneath the sentence, you find four ways of phrasing the underlined part. Following the requirements of standard written English, select the answer that produces the most effective sentence.\\
Tuberculosis, together with its effects, \underline{ranks one of the leading causes of death} in India.
\begin{enumerate}
    \item ranks as one of the leading causes of death
    \item rank as one of the leading causes of death
    \item has the rank of one of the leading causes of death
    \item are one of the leading causes of death
\end{enumerate}
\item Read the following paragraph and choose the correct statement.\\
    
    Climate change has reduced human security and threatened human well-being. An ignored reality of human progress is that human security largely depends upon environmental security. But on the contrary, human progress seems contradictory to environmental security. To keep up both at the required level is a challenge to be addressed by one and all. One of the ways to curb the climate change may be suitable scientific innovations, while the other may be the Gandhian perspective on small scale progress with focus on sustainability.
    
    \begin{enumerate}
        \item Human progress and security are positively associated with environmental security.
        \item Human progress is contradictory to environmental security.
        \item Human security is contradictory to environmental security.
        \item Human progress depends upon environmental security.
    \end{enumerate}
\item Fill the missing value \\
\begin{figure}[!ht]
\centering
\resizebox{0.5\textwidth}{!}{%
\begin{circuitikz}
\tikzstyle{every node}=[font=\large]
\draw  (4,12.25) circle (1cm);
\draw  (6,12.25) circle (1cm);
\draw  (8,12.25) circle (1cm);
\draw  (2,10.25) circle (1cm);
\draw  (4,10.25) circle (1cm);
\draw  (6,10.25) circle (1cm);
\draw  (8,10.25) circle (1cm);
\draw  (10,10.25) circle (1cm);
\draw  (2,8.25) circle (1cm);
\draw  (0,8.25) circle (1cm);
\draw  (10,8.25) circle (1cm);
\draw  (12,8.25) circle (1cm);
\draw  (6,8.25) circle (1cm);
\draw  (8,8.25) circle (1cm);
\draw  (4,8.25) circle (1cm);
\draw  (2,6.25) circle (1cm);
\draw  (4,6.25) circle (1cm);
\draw  (6,6.25) circle (1cm);
\draw  (8,6.25) circle (1cm);
\draw  (10,6.25) circle (1cm);
\draw  (4,4.25) circle (1cm);
\draw  (6,4.25) circle (1cm);
\draw  (8,4.25) circle (1cm);
\node [font=\large] at (4,12.25) {6};
\node [font=\large] at (6,12.25) {5};
\node [font=\large] at (8,12.25) {4};
\node [font=\large] at (2,10.25) {7};
\node [font=\large] at (4,10.25) {4};
\node [font=\large] at (6,10.25) {7};
\node [font=\large] at (8,10.25) {2};
\node [font=\large] at (10,10.25) {1};
\node [font=\large] at (4,4.25) {3};
\node [font=\large] at (8,4.25) {3};
\node [font=\large] at (0,8.25) {1};
\node [font=\large] at (2,8.25) {9};
\node [font=\large] at (4,8.25) {2};
\node [font=\large] at (6,8.25) {8};
\node [font=\large] at (8,8.25) {1};
\node [font=\large] at (10,8.25) {2};
\node [font=\large] at (12,8.25) {1};
\node [font=\large] at (2,.25) {4};
\node [font=\large] at (4,6.25) {1};
\node [font=\large] at (6,6.25) {5};
\node [font=\large] at (8,6.25) {2};
\node [font=\large] at (10,6.25) {3};
\end{circuitikz}
}%
\label{fig:my_label}
\end{figure}
\item A cube of side 3 units is formed using a set of smaller cubes of side 1 unit. Find the proportion of the number of faces of the smaller cubes visible to those which are NOT visible.
\begin{enumerate}
    \item 1:4
    \item 1:3
    \item 1:2
    \item 2:3
\end{enumerate}
\item Which one of the statements below is logically valid and can be inferred from the above sentences?
    \begin{enumerate}
        \item Humpty Dumpty always falls while having lunch.
        \item Humpty Dumpty does not fall sometimes while having lunch.
        \item Humpty Dumpty never falls during dinner.
        \item When Humpty Dumpty does not sit on the wall, the wall does not break.
    \end{enumerate}
    \newpage
\item Considering the matrix
    $\begin{pmatrix}
     0 & -1 & 2 \\ 1 & 0 & 3 \\ -2 & -3 & 0   
    \end{pmatrix}$
    which one of the following statements is \textbf{INCORRECT}?
    \begin{enumerate}
        \item One of its eigenvalues is zero.
        \item It has two purely imaginary eigenvalues.
        \item It has a non-zero real eigenvalue.
        \item The sum of its eigenvalues is zero.
    \end{enumerate}
\item The value of x where the function $f(x)=\sin(x)+\cos(x)$, defined over the domain $0\leq x\leq 2\pi$, attains a minimum is
\item The radius of convergence of the following power series is 
\\
    $\sum_{n=0}^{\infty}\frac{(x-3)^n}{3^n n!}$
\begin{enumerate}
    \item zero
    \item 1
    \item 3
    \item $\infty$
\end{enumerate}
