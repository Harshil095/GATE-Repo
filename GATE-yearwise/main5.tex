\iffalse
\title{Assignment5}
\author{ee24btech11064}
\chapter{2014}
\section{ae}
\fi

%\begin{enumerate}
\item A damped single degree of freedom system whose undamped natural frequency, $\omega_n = 10 \, \text{Hz}$, is subjected to sinusoidal external force. Power is half of the maximum for the two frequencies of $60.9469 \, \text{rad/s}$ and $64.7168 \, \text{rad/s}$. The damping factor associated with the vibrating system (in \%) is \underline{\hspace{4cm}}.
\vspace{0.5cm}
\item The boundary conditions for a rod with circular cross-section, under torsional vibration, are changed from fixed-free to fixed-fixed. The fundamental natural frequency of the fixed-fixed rod is \(k\) times that of the fixed-free rod. The value of \(k\) is
\begin{multicols}{2}
\begin{enumerate}
      \item $1.5$
        \item $\pi$
        \item $2.0$
        \item $0.5$
\end{enumerate}
\end{multicols}
\vspace{0.5cm}
\item Match the appropriate engine (in right column) with the corresponding aircraft (in left column) for most efficient performance of the engine.

\begin{itemize}
    \item[a.] Low speed transport aircraft \hspace{3cm} i. Ramjet
    \item[b.] High subsonic civilian aircraft \hspace{2.8cm} ii. Turboprop
    \item[c.] Supersonic fighter aircraft \hspace{3cm} iii. Turbojet
    \item[d.] Hypersonic aircraft \hspace{4.4cm} iv. Turbofan
\end{itemize}

\begin{itemize}
    \item[(A)] a -- iv, b -- iii, c -- i, d -- ii
    \item[(B)] a -- ii, b -- i, c -- iii, d -- iv
    \item[(C)] a -- i, b -- ii, c -- iv, d -- iii
    \item[(D)] a -- ii, b -- iv, c -- iii, d -- i
\end{itemize}

\vspace{0.5cm}
\item For a given fuel flow rate and thermal efficiency, the take-off thrust for a gas turbine engine burning aviation turbine fuel (considering fuel-air ratio $<1$) is

\begin{itemize}
    \item[(A)] Directly proportional to exhaust velocity
    \item[(B)] Inversely proportional to exhaust velocity
    \item[(C)] Independent of exhaust velocity
    \item[(D)] Directly proportional to the square of the exhaust velocity
\end{itemize}
\vspace{0.5cm}
\item For a fifty percent reaction axial compressor stage, following statements are given:
\begin{enumerate}
    \item[i.] Velocity triangles at the entry and exit of the rotor are symmetrical.
    \item[ii.] The whirl or swirl component of absolute velocity at the entry of rotor and entry of stator are the same.
\end{enumerate}
Which of the following options are correct?
\begin{itemize}
    \item[(A)] Both I and II are correct statements
    \item[(B)] I is correct but II is incorrect
    \item[(C)] I is incorrect but II is correct
    \item[(D)] Both I and II are incorrect
\end{itemize}

\vspace{0.5cm}
\item A small rocket having a specific impulse of 200 s produces a total thrust of 98 kN, out of which 10 kN is the pressure thrust. Considering the acceleration due to gravity to be $9.8 \, \text{m/s}^2$, the propellant mass flow rate in kg/s is
\begin{itemize}
    \item[(A)] 55.1
    \item[(B)] 44.9
    \item[(C)] 50
    \item[(D)] 60.2
\end{itemize}

\vspace{0.5cm}
\item The thrust produced by a turbojet engine
\begin{itemize}
    \item[(A)] Increases with increasing compressor pressure ratio
    \item[(B)] Decreases with increasing compressor pressure ratio
    \item[(C)] Remains constant with increasing compressor pressure ratio
    \item[(D)] First increases and then decreases with increasing compressor pressure ratio
\end{itemize}
\vspace{0.5cm}
\item The moment coefficient measured about the centre of gravity and about aerodynamic centre of a given wing-body combination are 0.0065 and -0.0235 respectively. The aerodynamic centre lies 0.06 chord lengths ahead of the centre of gravity. The lift coefficient for this wing-body is \underline{\hspace{2cm}}.
\vspace{0.5cm}

\item the vertical ground load factor on a stationary aircraft parked in its hangar is:
\begin{multicols}{2}
\begin{itemize}
    \item[(A)] 0
    \item[(B)] -1
    \item[(C)] Not defined
    \item[(D)] 1
\end{itemize}
\end{multicols}

\vspace{0.5cm}

\item Under what conditions should a glider be operated to ensure minimum sink rate?
\begin{multicols}{2}
\begin{itemize}
    \item[(A)] Maximum $C_L/C_D$
    \item[(B)] Minimum $C_L/C_D$
    \item[(C)] Maximum $C_D/C_L^{1/2}$
    \item[(D)] Minimum $C_D/C_L^{1/2}$
\end{itemize}
\end{multicols}

\vspace{0.5cm}

\item In most airplanes, the Dutch roll mode can be excited by applying
\begin{multicols}{2}
\begin{itemize}
    \item[(A)] A step input to the elevators
    \item[(B)] A step input to the rudder
    \item[(C)] A sinusoidal input to the ailerons
    \item[(D)] An impulse input to the elevators
\end{itemize}
\end{multicols}

\vspace{0.5cm}

\item Considering $\mathbf{R}$ as the radius of the moon, the ratio of the velocities of two spacecraft orbiting moon in circular orbit at altitudes $\mathbf{R}$ and $\mathbf{2R}$ above the surface of the moon is \underline{\hspace{2cm}}.
\vspace{0.5cm}

\item If $\myvec{A}=\myvec{3 & -3\\-3 & 4}$. Then det$\brak{-[A]^2+7[A]-3[I]}$ is
\begin{multicols}{2}
\begin{enumerate}
    \item $0$
    \item $-324$
    \item $324$
    \item $6$
\end{enumerate}
\end{multicols}
%\end{enumerate}
