\iffalse
	\title{2023-MA-53-65}
	\author{EE24Btech11006 - Arnav Mahishi}
	\section{ma}
	\chapter{2023}
\fi
\item{
Let $C\sbrak{0.1}=\cbrak{f:\sbrak{0,1}\rightarrow\mathbb{R}:f\text{ is continuous}}$ and $d_{\infty}\brak{f,g}=sup\cbrak{\abs{f\brak{x}-g\brak{x}}:x\in\sbrak{0,1}}$ for $f,g\in C\sbrak{0,1}$. For each $n\in\mathbb{N}$, define $f_n:\sbrak{0,1}\rightarrow\mathbb{R}$ by $f_n\brak{x}=x^n$ for all $x\in\sbrak{0,1}$. Let $P=\cbrak{f_n:n\in\mathbb{N}}$. Which of the following statements is/are correct? 
\begin{multicols}{2}
\begin{enumerate}
\item $P$ is totally bounded $\brak{C\sbrak{0,1},d_\infty}$
\item $P$ is bounded $\brak{C\sbrak{0,1},d_\infty}$
\item $P$ is closed $\brak{C\sbrak{0,1},d_\infty}$
\item $P$ is open$\brak{C\sbrak{0,1},d_\infty}$
\end{enumerate}
\end{multicols}
}
\item{
Let $G$ be an abelian group and $\phi:G\rightarrow\brak{\mathbb{Z},+}$ be a surjective homomorphism. Let $1=\phi\brak{a}$ for some $a\in G$.\\
Consider the following statements:\\
$P$: For every $g\in G$, there exists an $n\in\mathbb{Z}$ such that $ga^n\in ker\brak{\phi}$\\
$Q$: Let $e$ be the idendity of $G$ and $\langle a\rangle$ be the subgroup generated by $a$. Then $G=ker\brak{\phi}\langle a\rangle$ and $ker\brak{\phi}\cap\langle a\rangle=\cbrak{e}$.\\
Which of the following is/are correct?
\begin{multicols}{4}
\begin{enumerate}
\item $P$ is TRUE
\item $P$ is FALSE 
\item $Q$ is TRUE
\item $Q$ is FALSE
\end{enumerate}
\end{multicols}}
\item{
Let $C$ be the curve of intersection of the cylinder $x^2+y^2=4$ and the plane $z-2=0$. Suppose $C$ is oriented in the counterclockwise direction around the $z$-axis, when viewed from above. If $\abs{\int_c\brak{\sin x+e^x}dx+4xdy+e^z\cos^2zdz}=\alpha\pi$ then the value of $\alpha$ equals \rule{2cm}{0.15mm}
}
\item{
$l^2 = \cbrak{\brak{x_1, x_2, x_3, \ldots} : x_n \in \mathbb{R} \text{ for all } n \in \mathbb{N} \text{ and } \sum_{n=1}^{\infty} x_n^2 < \infty}$.
For a sequence $\brak{x_1, x_2, x_3, \ldots} \in l^2$, define $\norm{\brak{x_1, x_2, x_3, \ldots}}_2 = \sbrak{\sum_{n=1}^{\infty} x_n^2}^{\frac{1}{2}}$.
Consider the subspace $ M = \cbrak{\brak{x_1, x_2, x_3, \ldots} \in l^2 : \sum_{n=1}^{\infty} \frac{x_n}{4^n} = 0} $. Let $M^\perp$ denote the orthogonal complement of $M$ in the Hilbert space $\brak{l^2,\norm{.}_2}$. Consider $\brak{1,\frac{1}{2},\frac{1}{3},\frac{1}{4},\hdots}\in l^2$. If the orthogonal projection of $\brak{1,\frac{1}{2},\frac{1}{3},\frac{1}{4},\hdots}$ onto $M^\perp$ is given by $\alpha\brak{\sum_{n=1}^{\infty}\frac{1}{n4^n}}\brak{\rac{1}{4},\frac{1}{	4^2},\frac{1}{4^3},\hdots}$ for some $\alpha\in\mathbb{R}$, then $\alpha$ equals \rule{2cm}{0.15mm}\\
}
\item{
Consider the transportation problem between five sources and four destinations as given in the cost table below. The supply and demand at each of the source and destination are also provided:
\begin{table}[h!]    
  \centering
  \begin{table}[H]
    \centering
    \begin{tabular}{|c|c|c|c|c|c|c|}
        \hline
        & \multicolumn{4}{c|}{\textbf{DESTINATIONS}} & \textbf{Supply} \\ \cline{2-5}
        \textbf{SOURCES} & P & Q & R & S & \\ \hline
        \textbf{1} & 13 & 8 & 12 & 9 & 20 \\ \hline
        \textbf{2} & 10 & 7 & 5 & 20 & 10 \\ \hline
        \textbf{3} & 3 & 19 & 5 & 12 & 50 \\ \hline
        \textbf{4} & 4 & 9 & 7 & 15 & 30 \\ \hline
        \textbf{5} & 14 & 0 & 1 & 7 & 40 \\ \hline
        \textbf{Demand} & 60 & 10 & 20 & 60 & \\ \hline
    \end{tabular}
\end{table} 
  \caption{Input Parameters}
\end{table}

Let $C_N$ and $C_L$ be the total cost of the initial basic feasible solution obtained from the North-West corner method and the Least-Cost method, respectively. Then $C_N-C_L$ equals \rule{2cm}{0.15mm}
}
\item{
Let $\sigma\in S_8$, where $S_8$ is the permutation group on $8$ elements. Suppose $\sigma_1$ and $\sigma_2$, where $\sigma_1$ is the product of $\sigma_1$ and $\sigma_2$, where $\sigma_1$ is a $4$-cycle and $\sigma_2$ is a $3$-cycle in $S_8$. If $\sigma_1$ and $\sigma_2$ are disjoint cycles, then the number of elements in $S_8$ which are conjugate to $\sigma$ is \rule{2cm}{0.15mm}\\
}
\item{
Let $A$ be a $3\times 3$ real matrix with $det\brak{A+il}=0$, where $i=\sqrt{-1}$ and $I$ is the $3\times 3$ idendity matrix. If $det\brak{A}=3$, then the trace of $A^2$ is\rule{2cm}{0.15mm}\\
}
\item{
Let $A=\sbrak{a_{ij}}$ be a $3\times 3$ real matrix such that $A\myvec{1\\2\\1}=2\myvec{1\\2\\1},A\myvec{0\\1\\0}=2\myvec{0\\1\\1}$ and $A\myvec{-1\\1\\0}=4\myvec{-1\\1\\0}$ If $m$ is the degree of the minimal polynomial of $A$, then $a_{11}+a_{21}+a_{31}+m$ equals\rule{2cm}{0.15mm}.\\
}
\item{
Let $\Omega$ be the disk $x^2+y^2\textless 4$ in $\mathbb{R}^2$ with boundry $\delta\Omega$. If $u\brak{x,y}$ is the solution of the Dirichlet problem $\frac{\delta^2u}{\delta x^2}+\frac{\delta^2u}{\delta y^2}=0,\brak{x,y}\in\Omega$,\\$u\brak{x,y}=1+2x^2,\brak{x,y}\in\delta\Omega$ \\then the value of $u\brak{0,1}$ is \rule{2cm}{0.15mm}\\ 
}
\item{
For every $k\in\mathbb{N}\cup\cbrak{0}$, let $y_k\brak{x}$ be a polynomial of degree $k$ with $y_k\brak{1}=5$ Further, let $y_k\brak{x}$ satisfy the Lengendre equation $\brak{1-x^2}y^{\prime\prime}-2xy^{\prime}+k\brak{k+1}y=0$. If $\frac{1}{2}\int_{-1}^1\sum_{k=1}^n\brak{y_k\brak{x}-y_{k-1}\brak{x}}^2dx-\int_{-1}^1\sum_{k=1}^n\brak{y_k\brak{x}}^2dx=24$ for some positive integer $n$, then the value of $n$ is \rule{2cm}{0.15mm}\\
}
\item{
Consider the ordinary differential equation$\brak{\text{ODE}}$ $4\brak{lnx}y^{\prime\prime}+3y^{\prime}+y=0,x\textgreater 1$. If $r_1$ and $r_2$ are the roots of the indicial equation of the above ODE at the regular singular point $x=1$. Then $\abs{r_1-r_2}$ is equal to\rule{2cm}{0.15mm}$\brak{\text{rounded off to $2$ decimal places}}$\\
}
\item{
Let $u\brak{x,t}$ be the solution of the non-homogeneous wave equation\\ $\frac{\delta^2u}{\delta x^2}-\frac{\delta^2 u}{\delta t^2}=\sin x\sin\brak{2t},0\textless x\textless\pi,t\textgreater 0$,$u\brak{x,0}=0$, and $\frac{\delta u}{\delta t}\brak{x,0}=0$ for $0\leq x\leq\pi$, 
 $u\brak{0,t}=0,u\brak{\pi,t}=0$ for $t\geq 0$.
Then the value of $u\brak{\frac{\pi}{2},\frac{3\pi}{2}}$ is\rule{2cm}{0.15mm}$\brak{\text{rounded off to $2$ decimal places}}$
}
\item{
Consider the Linear Programming Problem $P$:\\ Maximize $3x_1+2x_2+5x_3$ subject to 
\begin{align}
    x_1+2x_2+x_3\leq 44\\
    x_1+2x_3\leq 48\\
    x_1+4x_2\leq 52,\\
    x_1\geq 0,x_2\geq 0, x_3\geq 0
\end{align}
The optimal value of the problem $P$ is \rule{2cm}{0.15mm}.
}
