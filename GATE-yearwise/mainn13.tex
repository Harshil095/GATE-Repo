\iffalse
\title{Assignment13}
\author{ee24btech11064}
\chapter{2023}
\section{ee}
\fi

%\begin{enumerate}
    \item Consider a unity-gain negative feedback system consisting of the plant $G(s)$(Given below) and a proportional-integral controller. Let the proportional gain and the integral gain be 3 and 1, respectively. For a unit step, reference input, the final values of the controller output and the plant output, respectively, are 
    \begin{align*}
        G(s)=\frac{1}{s-1}
    \end{align*}
    \begin{multicols}{2}
    \begin{enumerate}
            \item $\infty,\infty$
            \item 1,0
             \item 1,-1
            \item -1.1
    \end{enumerate}
    \end{multicols}
    \bigskip
\item The following columns present various modes of induction machine operation and the ranges of slip
\begin{center}
\begin{tabular}{|c|c|c|}
\hline
\multicolumn{1}{|c|}{\textbf{A}} & \multicolumn{1}{c|}{\textbf{B}} \\
\hline
\textbf{Mode of operation} &  \textbf{Range of Slip} \\
\hline
a.  Running in generator mode & p) From 0.0 to 1.0 \\
b. Running in motor mode     & q) From 1.0 to 2.0 \\
c.  Plugging in motor mode    & r) From -1.0 to 0.0 \\
\hline
\end{tabular}
\end{center}
The correct matching between the elements in column A with those of column B is
\begin{enumerate}
        \item a-r, b-p, and c-q
        \item a-r, b-q, and c-p
        \item a-p, b-r, and c-q
        \item a-q, b-p, and c-r
\end{enumerate}
\bigskip
\item A 10-pole, 50 Hz, 240 V, single phase induction motor runs at 540 RPM while driving rated load. Th frequency of induced rotor currents due to backward field is 
\begin{multicols}{2}
    \begin{enumerate}
        \item 100 Hz
        \item  95 Hz
        \item 10 Hz
        \item 5 Hz
    \end{enumerate}
\end{multicols}
\bigskip
\item A continuos-time system that is initially at rest is described by 
\begin{align*}
    \frac{dy(t)}{dt}+3y(t)=2x(t)
\end{align*}
Where x(t) is the input voltage and y(t) is the output voltage. The impulse response of the system is 
\begin{enumerate}
        \item $3e^{-2t}$
        \item $\frac{1}{3}e^{-2t}u(t)$
        \item $2e^{-3t}u(t)$
        \item $2e^{-3t}$
\end{enumerate}
\bigskip
\item The Fourier transform $X(\omega)$ of the signal $x(t)$ is given by
\[
X(\omega) =
\begin{cases}
1, & \text{for } |\omega| < W_0 \\
0, & \text{for } |\omega| > W_0
\end{cases}
\]
Which one of the following statements is true?
\begin{enumerate}
    \item $x(t)$ tends to be an impulse as $W_0 \to \infty$.
    \item $x(0)$ decreases as $W_0$ increases.
    \item At $t = \frac{\pi}{2W_0}$, $x(t) = -\frac{1}{\pi}$.
    \item At $t = \frac{\pi}{2W_0}$, $x(t) = \frac{1}{\pi}$.
\end{enumerate}
\bigskip
\item The Z-transform of a discrete signal x[n] is 
\begin{align*}
    X(z)=\frac{4z}{(z-\frac{1}{5})(z-\frac{2}{3})(z-3)} \text{with ROC = R.}
\end{align*}
Which one of the following statements is true?
\begin{enumerate}

        \item Discrete-time Fourier transform of x[n] converges if R is $|z|>3$
        \item Discrete-time Fourier transform of x[n] converges if R is $\frac{2}{3}<|z|<3$
        \item Discrete-time Fourier transform of x[n] converges if R is such that x[n] is a left-sided sequence
        \item Discrete-time Fourier transform of x[n] converges if R is such that x[n] is a right-sided sequence
\end{enumerate}
\bigskip
\item For the three-bus power system shown in the figure, the trip signals to the circuit breakers $B_1$ to $B_9$ are provided by over current relays $R_1$ to $R_9$, respectively, some of which have directional properties also. The necessary condition for the system to be protected for short circuit fault at any part of the system between the bus 1 and the R-L loads with isolation of minimum portion of the network using minimum number of directional relays is
\begin{figure}[H]
\centering
\resizebox{0.8\textwidth}{!}{%
\begin{circuitikz}
\tikzstyle{every node}=[font=\normalsize]
% Draw a single sinusoidal voltage source
\draw (3.25,8.75) to[sinusoidal voltage source, sources/symbol/rotate=auto] (4.5,8.75);
% Connect the source to B_9
\draw (4.5,8.75) -- (5.25,8.75) node[above] {$B_9$};

% Rest of the circuit
\draw (5.25,8.75) to[short] (5.75,8.75);
\draw (5.75,8.75) to[short] (5.75,9.75);
\draw (5.75,8.75) to[short] (5.75,7.75);
\draw (5.75,9.5) to[short, -o] (6.25,9.5) node[above] {$B_1$};
\draw (5.75,8) to[short, -o] (6.25,8) node[below] {$B_2$};
\draw [short] (6.25,9.5) -- (9,9.5);
\draw [short] (6.25,8) -- (9,8);
\draw (9,9.5) to[short, -o] (9.75,9.5) node[above] {$B_3$};
\draw (9,8) to[short, -o] (9.75,8) node[below] {$B_4$};
\draw [short] (9.75,9.5) -- (10.25,9.5);
\draw [short] (9.75,8) -- (10.25,8);
\draw [short] (10.25,9.75) -- (10.25,7.75);
\draw (10.25,9.25) to[short, -o] (10.75,9.25) node[above] {$B_5$};
\draw (10.25,7.75) to[short, -o] (10.75,7.75) node[above] {$B_6$};
\draw [short] (10.75,9.25) -- (12.75,9.25);
\draw [short] (10.75,7.75) -- (11.5,7.75);
\draw [->, >=Stealth] (11.5,7.75) -- (11.5,7);
\draw (12.75,9.25) to[short, -o] (13,9.25) node[above] {$B_7$};
\draw [short] (13,9.25) -- (13.5,9.25);
\draw [short] (13.5,9.75) -- (13.5,8.75);
\draw (13.5,9.25) to[short, -o] (14,9.25) node[above] {$B_8$};
\draw [short] (14,9.25) -- (14.75,9.25);
\draw [->, >=Stealth] (14.75,9.25) -- (14.75,8.5);
\draw  (5.75,10.25) circle (0.25cm);
\draw  (10.25,10.25) circle (0.25cm);
\draw  (13.5,10.5) circle (0.25cm);
\node [font=\normalsize] at (5.75,10.25) {1};
\node [font=\normalsize] at (10.25,10.25) {2};
\node [font=\normalsize] at (13.5,10.5) {3};
\node [font=\normalsize] at (12,7) {$R-L$};
\node [font=\normalsize] at (15.25,8.5) {$R-L$};
\node [font=\normalsize] at (12,6.5) {$load$};
\node [font=\normalsize] at (15,8) {$load$};
\node [font=\normalsize] at (7.5,9.75) {$line 1$};
\node [font=\normalsize] at (7.5,7.75) {$line 2$};
\node [font=\normalsize] at (11.75,9.5) {$line 3$};
\end{circuitikz}
}%
\end{figure}

\begin{enumerate}
    \item $R_3$ and $R_4$ are directional over current relays blocking faults towards bus 2
    \item $R_3$ and $R_4$ are directional over current relays blocking faults towards bus 2 and $R_7$ is directional over current relay blocking faults towards bus 3
    \item $R_3$ and $R_4$ are directional over current relays blocking faults towards line 1 and line 2, respectively, $R_7$ is directional over current relay blocking faults towards line 3 and $R_5$ is directional over current relay blocking faults towards bus 2 
    \item $R_3$ and $R_4$ are directional over current relays blocking faults towards line 1 and line 2, respectively.

\end{enumerate}
\bigskip
\item The expressions of fuel cost of two thermal generating units as a function of the respective power generation $P_{G1}$ and $P_{G2}$ are given as
\[
F_1(P_{G1}) = 0.1a P_{G1}^2 + 40 P_{G1} + 120 \text{ Rs/hour} \quad 0 \text{ MW} \leq P_{G1} \leq 350 \text{ MW}
\]
\[
F_2(P_{G2}) = 0.2 P_{G2}^2 + 30 P_{G2} + 100 \text{ Rs/hour} \quad 0 \text{ MW} \leq P_{G2} \leq 300 \text{ MW}
\]
where $a$ is a constant. For a given value of $a$, optimal dispatch requires the total load of 290 MW to be shared as $P_{G1} = 175 \text{ MW}$ and $P_{G2} = 115 \text{ MW}$. With the load remaining unchanged, the value of $a$ is increased by 10\% and optimal dispatch is carried out. The changes in $P_{G1}$ and the total cost of generation, $F \,(= F_1 + F_2)$ in Rs/hour will be as follows:
\begin{enumerate}

        \item $P_{G1}$ will decrease and F will increase
        \item Both $P_{G1}$ and F will increase
        \item $P_{G1}$ will increase and F will decrease
        \item Both $P_{G1}$ and F will decrease
    
\end{enumerate}
\bigskip
\item The four stator conductors (A, A', B and B') of a rotating machine are carrying DC currents of the same value, the directions of which are shown in the figure (i). The rotor coils \( a-a' \) and \( b-b' \) are formed by connecting the back ends of conductors 'a' and 'a'' and 'b' and 'b'', respectively, as shown in figure (ii). The e.m.f. induced in coil \( a-a' \) and coil \( b-b' \) are denoted by \( E_{a-a'} \) and \( E_{b-b'} \), respectively. If the rotor is rotated at uniform angular speed \( \omega \) rad/s in the clockwise direction then which of the following correctly describes the \( E_{a-a'} \) and \( E_{b-b'} \)?
\begin{figure}[H]
\centering
\resizebox{0.5\textwidth}{!}{%
\begin{circuitikz}
\tikzstyle{every node}=[font=\normalsize]
\draw  (7.25,8.25) circle (4cm);
\draw [short] (4.25,8) .. controls (4.25,9.75) and (5.75,11.25) .. (7,11);
\draw [short] (7,11) -- (7,11.5);
\draw [short] (7,11.5) -- (7.5,11.5);
\draw [short] (7.5,11.5) -- (7.5,11);
\draw [short] (4.25,8) -- (3.75,8);
\draw [short] (3.75,8) -- (3.75,7.5);
\draw [short] (3.75,7.5) -- (4.25,7.5);
\draw [short] (4.25,7.5) .. controls (4.75,6) and (6,5.5) .. (7,5.25);
\draw [short] (7,5.25) -- (7,4.75);
\draw [short] (7,4.75) -- (7.5,4.75);
\draw [short] (7.5,4.75) -- (7.5,5.25);
\draw [short] (7.5,11) .. controls (9.25,10.75) and (10,9.5) .. (10,8.25);
\draw [short] (7.5,5.25) .. controls (9.5,5.5) and (10,7) .. (10,7.75);
\draw [short] (10,8.25) -- (10.5,8.25);
\draw [short] (10.5,8.25) -- (10.5,7.75);
\draw [short] (10.5,7.75) -- (10,7.75);
\draw [short] (5,8.25) .. controls (5.25,9.75) and (6,10.25) .. (7,10.25);
\draw [short] (5,8.25) -- (5.5,8.25);
\draw [short] (5.5,8.25) -- (5.5,7.75);
\draw [short] (5,7.75) -- (5.5,7.75);
\draw [short] (7,10.25) -- (7,9.75);
\draw [short] (7,9.75) -- (7.5,9.75);
\draw [short] (7.5,9.75) -- (7.5,10.25);
\draw [short] (5,7.75) .. controls (5.5,6.5) and (6,6) .. (7,6);
\draw [short] (7.5,10.25) .. controls (8.75,10.25) and (9.25,9.25) .. (9.25,8.25);
\draw [short] (7.5,6) .. controls (8.75,6) and (9.25,7) .. (9.25,7.75);
\draw [short] (7,6) -- (7,6.5);
\draw [short] (7,6.5) -- (7.5,6.5);
\draw [short] (7.5,6.5) -- (7.5,6);
\draw [short] (8.75,7.75) -- (9.25,7.75);
\draw  (7.25,11.25) circle (0.25cm);
\draw  (7.25,5) circle (0.25cm);
\draw  (9,8) circle (0.25cm);
\draw  (7.25,10) circle (0.25cm);
\draw  (10.25,8) circle (0.25cm);
\draw  (4,7.75) circle (0.25cm);
\draw  (5.25,8) circle (0.25cm);
\draw  (7.25,6.25) circle (0.25cm);
\draw [short] (7.25,11.5) -- (7.25,11);
\draw [short] (7,11.25) -- (7.5,11.25);
\draw [short] (7.25,5.25) -- (7.25,4.75);
\draw [short] (7,5) -- (7.5,5);
\node [font=\Huge] at (4,7.75) {\textbf{.}};
\node [font=\Huge] at (4,8) {\textbf{.}};
\node [font=\Huge] at (10.25,8.25) {\textbf{.}};
\node [font=\Huge] at (10.25,8) {\textbf{.}};
\draw [short] (8.75,7.75) -- (8.75,8.25);
\draw [short] (8.75,8.25) -- (9.25,8.25);
\draw [->, >=Stealth] (5.25,9.75) .. controls (6.75,11.25) and (8.25,10.75) .. (8.75,10) ;
\node [font=\normalsize] at (9,10) {$\omega$};
\node [font=\normalsize] at (7.25,9.5) {a};
\node [font=\normalsize] at (8.5,8) {b};
\node [font=\normalsize] at (7.25,6.75) {a'};
\node [font=\normalsize] at (5.75,8) {b'};
\node [font=\normalsize] at (6.5,11.5) {A};
\node [font=\normalsize] at (7.75,5) {B};
\node [font=\normalsize] at (10.5,8.5) {A'};
\node [font=\normalsize] at (3.75,8.5) {B'};
\end{circuitikz}
}%

\caption{figure (i): Cross-sectional view}
\end{figure}

\begin{figure}[H]
\centering
\resizebox{0.5\textwidth}{!}{%
\begin{circuitikz}
\tikzstyle{every node}=[font=\normalsize]
\draw [short] (3.5,9.25) -- (3.5,7);
\draw [short] (3.5,9.25) -- (4.75,10);
\draw [short] (4.75,10) -- (6,9.25);
\draw [short] (6,9.25) -- (6,7);
\draw [short] (3.5,7) -- (4.5,6.25);
\draw [short] (6,7) -- (5.25,6.25);
\draw [short] (7.25,9.25) -- (7.25,7);
\draw [short] (7.25,9.25) -- (8.5,10);
\draw [short] (8.5,10) -- (9.75,9.25);
\draw [short] (9.75,9.25) -- (9.75,7);
\draw [short] (7.25,7) -- (8.25,6.25);
\draw [short] (9.75,7) -- (9,6.25);
\draw (4.5,6.25) to[short, -o] (4.5,5.5);
\draw (5.25,6.25) to[short, -o] (5.25,5.5);
\draw (8.25,6.25) to[short, -o] (8.25,5.5);
\draw (9,6.25) to[short, -o] (9,5.5);
\node [font=\normalsize] at (3.25,8.25) {a};
\node [font=\normalsize] at (6.25,8.25) {a'};
\node [font=\normalsize] at (7,8.25) {b};
\node [font=\normalsize] at (10,8.25) {b'};
\end{circuitikz}
}
\caption{figure (ii): rotor winding connection diagram}
\end{figure}
\begin{enumerate}
    \item $E_{a-a'}$ and $E_{b-b'}$ have finite magnitudes and are in same phase 
    \item $E_{a-a'}$ and $E_{b-b'}$ have finite magnitudes with $E_{b-b'}$ leading $E_{a-a'}$
    \item $E_{a-a'}$ and $E_{b-b'}$ have finite magnitudes with $E_{a-a'}$ leading $E_{b-b'}$
    \item $E_{a-a'}=E_{b-b'}=0$
\end{enumerate}

\bigskip
\item The chopper circuit shown in figure (i) feeds power to a 5 A DC constant current source. The switching frequency of the chopper is 100kHz. All the components can be assumed to be ideal. The gate signals of switches $S_1$ and $S_2$ are shown in the figure (ii). Average voltage across the 5A current source is 
\begin{figure}[H]
\centering
\resizebox{0.5\textwidth}{!}{%
\begin{circuitikz}
\tikzstyle{every node}=[font=\normalsize]
\draw (4,9.75) to[battery1] (4,7.25);
\draw [short] (4,9.75) -- (6,9.75);
\draw [short] (4,7.25) -- (4,5);
\draw [short] (4,5) -- (6,5);
\draw (6,7.75) to[Tnpn, transistors/scale=1.19] (6,9.75);
\draw (6,5) to[Tnpn, transistors/scale=1.19] (6,7);
\draw [short] (6,7) -- (6,7.75);
\draw [short] (6,7.5) -- (8.75,7.5);
\draw (8.75,7.5) to[american current source] (8.75,5.25);
\draw [short] (6,5) -- (8.75,5);
\draw [short] (8.75,5) -- (8.75,5.25);
\draw (6.75,8.25) to[D] (6.75,9.25);
\draw (6.75,5.5) to[D] (6.75,6.5);
\draw [short] (6,9.25) -- (6.75,9.25);
\draw [short] (6,8.25) -- (6.75,8.25);
\draw [short] (6,6.5) -- (6.75,6.5);
\draw [short] (6,5.5) -- (6.75,5.5);
\node [font=\normalsize] at (3.25,8.5) {20 V};
\node [font=\normalsize] at (4.25,8.75) {+};
\node [font=\normalsize] at (5.75,8.75) {$S_1$};
\node [font=\normalsize] at (5.75,6) {$S_2$};
\node [font=\normalsize] at (7.25,8.75) {$D_1$};
\node [font=\normalsize] at (7.25,6) {$D_2$};
\node [font=\normalsize] at (9.5,6.5) {5 A};
\end{circuitikz}
}%

\caption{figure (i)}
\end{figure}

\begin{figure}[H]
\centering
\resizebox{0.5\textwidth}{!}{%
\begin{circuitikz}
\tikzstyle{every node}=[font=\normalsize]
\draw [short] (3,10.5) -- (3,6);
\draw [short] (3,6) -- (7.75,6);
\draw [->, >=Stealth] (7.75,6) -- (9.5,6);
\draw [short] (3,8.25) -- (9.25,8.25);
\draw [dashed] (4,6) -- (4,8.25);
\draw [dashed] (6.75,6) -- (6.75,8.25);
\draw [line width=1pt, short] (4,8.25) -- (4,9.5);
\draw [line width=1pt, short] (3,9.5) -- (4,9.5);
\draw [line width=1pt, short] (4,8.25) -- (6.75,8.25);
\draw [line width=1pt, short] (6.75,8.25) -- (6.75,9.5);
\draw [line width=1pt, short] (6.75,9.5) -- (7.5,9.5);
\draw [line width=1pt, short] (7.5,9.5) -- (7.75,9.5);
\draw [line width=1pt, short] (7.75,9.5) -- (7.75,8.25);
\draw [line width=1pt, short] (7.75,8.25) -- (9.25,8.25);
\draw [line width=1pt, short] (3,6) -- (4.5,6);
\draw [line width=1pt, short] (4.5,6) -- (4.5,7);
\draw [line width=1pt, short] (4.5,7) -- (4.5,7.25);
\draw [line width=1pt, short] (4.5,7.25) -- (5.5,7.25);
\draw [line width=1pt, short] (5.5,7.25) -- (5.5,6);
\draw [line width=1pt, short] (5.5,6) -- (7.75,6);
\draw [line width=1pt, short] (7.75,6) -- (8,6);
\draw [line width=1pt, short] (8,6) -- (8,7.25);
\draw [line width=1pt, short] (8,7.25) -- (8.75,7.25);
\draw [line width=1pt, short] (8.75,7.25) -- (8.75,6);
\draw [line width=1pt, short] (8.75,6) -- (9.25,6);
\node [font=\normalsize] at (3,5.75) {0};
\node [font=\normalsize] at (4,5.75) {3};
\node [font=\normalsize] at (4.5,5.75) {5};
\node [font=\normalsize] at (5.5,5.75) {8};
\node [font=\normalsize] at (6.75,5.75) {10};
\node [font=\normalsize] at (8.5,5.75) {time in $\mu$ s};
\node [font=\normalsize] at (5,8.5) {Low};
\node [font=\normalsize] at (3.5,9.75) {High};
\node [font=\normalsize] at (2.25,9.5) {Gate};
\node [font=\normalsize] at (2.25,7.25) {Gate};
\node [font=\normalsize] at (2.25,9) {Signal};
\node [font=\normalsize] at (2.25,6.75) {Signal};
\node [font=\normalsize] at (2.25,8.5) {of $S_1$};
\node [font=\normalsize] at (2.25,6.25) {of $S_2$};
\end{circuitikz}
}%

\caption{figure(ii)}
\end{figure}
\begin{multicols}{2}
    \begin{enumerate}
        \item 10 V
        \item 6 V 
        \item 12 V
        \item 20 V
    \end{enumerate}
\end{multicols}
\bigskip
\item In the figure, the vectors \textbf{u} and \textbf{v} are related as: $Au=v$ by a transformation matrix \textbf{A}. The correct choice of \textbf{A} is 
\begin{figure}[H]
\centering
\resizebox{0.4\textwidth}{!}{%
\begin{circuitikz}
\tikzstyle{every node}=[font=\normalsize]
\draw [line width=1.5pt, ->, >=Stealth] (4.25,5.75) -- (10.75,5.75);
\draw [line width=1.5pt, ->, >=Stealth] (4.25,5.75) -- (9,10.25);
\node [font=\normalsize] at (3.75,5.5) {(0,0)};
\node [font=\normalsize] at (10.5,5.25) {(5,0)};
\node [font=\normalsize] at (8.5,10.5) {(4,3)};
\node [font=\normalsize] at (9.25,10.25) {u};
\node [font=\normalsize] at (11,5.75) {v};
\end{circuitikz}
}%
\end{figure}
\begin{multicols}{2}
\begin{enumerate}
    \item $\myvec{\frac{4}{5} & \frac{3}{5}\\ 
    -\frac{3}{5}&\frac{4}{5}}$
    \item $\myvec{\frac{4}{5} & -\frac{3}{5}\\ 
    \frac{3}{5}&\frac{4}{5}}$
    \item $\myvec{\frac{4}{5} & \frac{3}{5}\\ 
    \frac{3}{5}&\frac{4}{5}}$
    \item $\myvec{\frac{4}{5} & -\frac{3}{5}\\ 
    \frac{3}{5}& -\frac{4}{5}}$
\end{enumerate}
\end{multicols}
\bigskip
\item One million random numbers are generated from a statistically stationary process with a Gaussian distribution with mean zero and standard deviation $\sigma_o$.\\
The $\sigma_o$ is estimated by randomly drawing out 10,000 numbers of samples ($x_n$). The estimates $\hat{\sigma_1}$, $\hat{\sigma_2}$ are computed in the following two ways.
\begin{align*}
    \hat{\sigma_1}^2 &= \frac{1}{10000}\sum_{n=1}^{10000}x_n ^2\\
    \hat{\sigma_2}^2 &= \frac{1}{9999}\sum_{n=1}^{10000}x_n ^2
\end{align*}
Which of the following statements is true?
\begin{multicols}{2}
\begin{enumerate}
    \item $E( \hat{\sigma_2}^2 )= \hat{\sigma_o}^2 $
    \item $E(\hat{\sigma_2})=\sigma_o$
    \item $E( \hat{\sigma_1}^2 )= \hat{\sigma_o}^2 $
    \item $E(\hat{\sigma_1}) = E(\hat{\sigma_2})$
\end{enumerate}
\end{multicols}
\bigskip
\item A semiconductor switch needs to block voltage V of only one polarity ($V>0$) during OFF state as shown in figure(i) and carry current in both directions during ON state as shown in figure(ii). Which of the following switch combinations will realize the same?
\begin{figure}[H]
\centering
\resizebox{0.5\textwidth}{!}{%
\begin{circuitikz}
\tikzstyle{every node}=[font=\normalsize]
\draw [ line width=0.9pt](3,8) to[short, -o] (4.5,8) ;
\draw [ line width=0.9pt](7.5,8) to[short, -o] (6,8) ;
\draw [line width=0.9pt, ->, >=Stealth] (4.5,8) -- (5.5,8.75);
\node [font=\normalsize] at (2.75,8) {P};
\node [font=\normalsize] at (7.75,8) {Q};
\node [font=\normalsize] at (5.25,7.75) {V};
\node [font=\normalsize] at (4.5,7.75) {+};
\node [font=\normalsize] at (5.75,7.75) {-};
\end{circuitikz}
}%
\caption{figure(i)}
\end{figure}
\begin{figure}[H]
\centering
\resizebox{0.4\textwidth}{!}{%
\begin{circuitikz}
\tikzstyle{every node}=[font=\normalsize]
\draw [ line width=0.9pt](3,8) to[short, -o] (4.5,8) ;
\draw [ line width=0.9pt](7.5,8) to[short, -o] (6,8) ;
\draw [line width=0.9pt, ->, >=Stealth] (4.5,8) -- (6,8.25);
\node [font=\normalsize] at (2.75,8) {P};
\node [font=\normalsize] at (7.75,8) {Q};
\draw [line width=0.9pt, ->, >=Stealth] (4.75,7.5) -- (5.5,7.5);
\draw [line width=0.9pt, ->, >=Stealth] (5.5,7.25) -- (4.75,7.25);
\draw [line width=0.9pt, short] (5.75,7.25) -- (5.75,7.75);
\draw [line width=0.9pt, short] (5.75,7.25) -- (5.75,7);
\end{circuitikz}
}%

\caption{figure(ii)}
\end{figure}
\begin{enumerate}
    \item \begin{figure}[H]
\centering
\resizebox{0.4\textwidth}{!}{%
\begin{circuitikz}
\tikzstyle{every node}=[font=\large]
\draw [line width=0.9pt](5.5,8.25) to[Tnpn, transistors/scale=1.19] (3,8.25);
\draw [ line width=0.9pt](5.5,9.25) to[D] (3,9.25);
\draw [line width=0.9pt, short] (3,8.25) -- (2.5,8.25);
\draw [line width=0.9pt, short] (3,9.25) -- (3,8.25);
\draw [line width=0.9pt, short] (5.25,8.25) -- (6,8.25);
\draw [line width=0.9pt, short] (5.5,9.25) -- (5.5,8.25);
\node [font=\large] at (2,8.25) {P};
\node [font=\large] at (6.5,8.25) {Q};
\end{circuitikz}
}%
\end{figure}
\item \begin{figure}[H]
\centering
\resizebox{0.4\textwidth}{!}{%
\begin{circuitikz}
\tikzstyle{every node}=[font=\large]
\draw [line width=0.9pt](5.5,8.25) to[Tnpn, transistors/scale=1.19] (3,8.25);
\draw [ line width=0.9pt](5.5,9.25) to[D] (3,9.25);
\draw [line width=0.9pt, short] (3,8.25) -- (2.5,8.25);
\draw [line width=0.9pt, short] (3,9.25) -- (3,8.25);
\draw [line width=0.9pt, short] (5.25,8.25) -- (6,8.25);
\draw [line width=0.9pt, short] (5.5,9.25) -- (5.5,8.25);
\node [font=\large] at (2,8.25) {P};
\draw [ line width=0.9pt](7.5,8.25) to[D] (6,8.25);
\node [font=\large] at (8,8.25) {Q};
\end{circuitikz}
}%
\end{figure}
\item \begin{figure}[H]
\centering
\resizebox{0.4\textwidth}{!}{%
\begin{circuitikz}
\tikzstyle{every node}=[font=\large]
\draw [line width=0.9pt](5.5,8.25) to[Tnpn, transistors/scale=1.19] (3,8.25);
\draw [ line width=0.9pt](5.5,9.25) to[D] (3,9.25);
\draw [line width=0.9pt, short] (3,8.25) -- (2.5,8.25);
\draw [line width=0.9pt, short] (3,9.25) -- (3,8.25);
\draw [line width=0.9pt, short] (5.25,8.25) -- (6,8.25);
\draw [line width=0.9pt, short] (5.5,9.25) -- (5.5,8.25);
\node [font=\large] at (2,8.25) {P};
\node [font=\large] at (8,8.25) {Q};
\draw [ line width=0.9pt](6,8.25) to[D] (7.5,8.25);
\end{circuitikz}
}%
\end{figure}
\item \begin{figure}[H]
\centering
\resizebox{0.4\textwidth}{!}{%
\begin{circuitikz}
\tikzstyle{every node}=[font=\large]
\draw [line width=0.9pt](5.5,8.25) to[Tnpn, transistors/scale=1.19] (3,8.25);
\draw [ line width=0.9pt](5.5,9.25) to[D] (3,9.25);
\draw [line width=0.9pt, short] (3,8.25) -- (2.5,8.25);
\draw [line width=0.9pt, short] (3,9.25) -- (3,8.25);
\draw [line width=0.9pt, short] (5.25,8.25) -- (6,8.25);
\draw [line width=0.9pt, short] (5.5,9.25) -- (5.5,8.25);
\node [font=\large] at (2,8.25) {P};
\node [font=\large] at (8,8.25) {Q};
\draw [ line width=0.9pt](6,8.25) to[D] (7.5,8.25);
\draw [line width=0.9pt, short] (7.25,8.25) -- (7.25,10);
\draw [line width=0.9pt, short] (2.75,8.25) -- (2.75,10);
\draw [ line width=0.9pt](7.25,10) to[D] (2.75,10);
\end{circuitikz}
}%

\end{figure}
\end{enumerate}
%\end{enumerate}
