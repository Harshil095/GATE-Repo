\iffalse
	\title{2024-ST-1-13}
	\author{EE24Btech11006 - Arnav Mahishi}
	\section{st}
	\chapter{2024}
\fi
\item{
 If $\rightarrow$ denotes increasing order of intensity, then the meaning of the words $\sbrak{walk\rightarrow jog\rightarrow sprint}$ is analogous to $bothered\rightarrow\rule{2cm}{0.15mm}\rightarrow daunted$. Which one of the given options is appropriate to fill the blank?
\begin{multicols}{4}
\begin{enumerate}
\item phased
\item phrased
\item fazed
\item fused
\end{enumerate}
\end{multicols}
}
\item{
Two wizards try to create a spell using all the four elements, water, air, fire, and earth. For this, they decide to mix all these elements in all possible orders. They also decide to work independently. After trying all possible combination of elements, they conclude that the spell does not work.
How many attempts does each wizard make before coming to this conclusion, independently?
\begin{multicols}{4}
\begin{enumerate}
\item $24$ 
\item $48$ 
\item $16$
\item $12$
\end{enumerate}
\end{multicols}}
\item{
In an engineering college of $10000$ students, $1500$ like neither their core branches nor other branches. The number of students who like their core branches is $\frac{1}{4}$th of the number of students who like other branches. The number of students who like both their core and other branches is $500$.
The number of students who like their core branches is
\begin{multicols}{4}
\begin{enumerate}
\item $1800$
\item $3500$
\item $1600$
\item $1500$
\end{enumerate}
\end{multicols}
}
\item{
For positive non-zero real variables $x$ and $y$, if $ln\brak{\frac{x+y}{2}}=\frac{1}{2}\sbrak{ln\brak{x}+ln\brak{y}}$ then the value of $\frac{x}{y}+\frac{y}{x}$ is
\begin{multicols}{4}
\begin{enumerate}
\item $1$
\item $\frac{1}{2}$
\item $2$
\item $4$
\end{enumerate}
\end{multicols}
}
\item{
In the sequence $6,9,14,x,30,41$, a possible value of $x$ is
\begin{multicols}{4}
\begin{enumerate}
\item $25$
\item $21$
\item $18$
\item $20$
\end{enumerate}
\end{multicols}
}
\item{
Sequence the following sentences in a coherent passage.\\
P: This fortuitous geological event generated a colossal amount of energy and heat that resulted in the rocks rising to an average height of $4$ km across the contact zone.\\
Q: Thus, the geophysicists tend to think of the Himalayas as an active geological event rather than as a static geological feature.\\
R: The natural process of the cooling of this massive edifice absorbed large quantities of atmospheric carbon dioxide, altering the earth's atmosphere and making t better suited for life.\\
S: Many millennia ago, a breakaway chunk of bedrock from the Antarctic Plate collided with the massive Eurasian Plate.\\
\begin{multicols}{4}
\begin{enumerate}
\item QPSR
\item QSPR
\item SPRQ
\item SRPQ
\end{enumerate}
\end{multicols}
}
\item{
A person sold two different items at the same price. He made $10\%$ profit in one item, and $10\%$ loss in the other item. In selling these two items, the person made a total of
\begin{multicols}{4}
\begin{enumerate}
\item $1\%$ profit
\item $2\%$ profit
\item $1\%$ loss
\item $2\%$ loss
\end{enumerate}
\end{multicols}
}
\item{
The pie charts depict the shares of various power generation technologies in the total electricity generation of a country for the years 2007 and 2023.
\begin{figure}[H]
\centering
\resizebox{10cm}{!}{%
\begin{circuitikz}
\tikzstyle{every node}=[font=\Large]
\draw  (5.25,9.5) circle (2cm);
\draw  (11.75,9.5) circle (2cm);
\draw [short] (5.25,9.5) -- (5.5,11.5);
\draw [short] (5.25,9.5) -- (4.75,11.5);
\draw [short] (5.25,9.5) -- (3.75,8);
\draw [short] (5.25,9.5) -- (4.25,7.75);
\draw [short] (5.25,9.5) -- (6.75,8);
\draw [short] (11.75,9.5) -- (11.75,11.5);
\draw [short] (11.75,9.5) -- (10,10.5);
\draw [short] (11.75,9.5) -- (13.25,10.75);
\draw [short] (11.75,9.5) -- (12.5,7.75);
\draw [short] (11.75,9.5) -- (13.5,8.5);
\node [font=\normalsize] at (4.25,9.75) {Hydro 30\%};
\node [font=\normalsize] at (3.5,7.5) {Wind 5\%};
\node [font=\normalsize] at (5.5,8.5) {Gas 25\%};
\node [font=\normalsize] at (6.5,9.75) {Coal 35\%};
\node [font=\normalsize] at (5,12) {Solar 5\%};
\node [font=\normalsize] at (11,9) {Hydro 35\%};
\node [font=\normalsize] at (11,10.5) {Solar 20\%};
\node [font=\normalsize] at (12.5,10.75) {Coal 20\%};
\node [font=\normalsize] at (13,9.5) {Gas 15\%};
\node [font=\normalsize] at (13.5,7.5) {Wind 10\%};
\node [font=\Large] at (5,13) {Year 2007};
\node [font=\Large] at (11.5,13) {Year 2023};
\draw [->, >=Stealth] (4.25,8.25) -- (3.75,7.75);
\draw [->, >=Stealth] (12.75,8.75) -- (13.5,8);
\draw [->, >=Stealth] (5,11) -- (5,11.75);
\end{circuitikz}
}%

\label{fig:my_label}
\end{figure}
The renewable sources of electricity generation consist of Hydro, Solar and Wind. Assuming that the total electricity generated remains the same from 2007 to 2023, what is the percentage increase in the share of the renewable sources of electricity generation over this period?
\begin{multicols}{4}
\begin{enumerate}
\item $25\%$
\item $50\%$
\item $77.5\%$
\item $62.5\%$
\end{enumerate}
\end{multicols}
}
\item{
A cube is to be cut into $8$ pieces of equal size and shape. Here, each cut should be straight and it should not stop till it reaches the other end of the cube.The minimum number of such cuts required is
\begin{multicols}{4}
\begin{enumerate}
\item $3$
\item $4$
\item $7$ 
\item $8$
\end{enumerate}
\end{multicols}
}
\item{
In the $4\times 4$ array shown below, each cell of the first three rows has either a cross$\brak{\times}$ or a number.
\begin{figure}[H]
\centering
\resizebox{2cm}{!}{%
\begin{circuitikz}
\tikzstyle{every node}=[font=\LARGE]
\draw  (6.25,11.75) rectangle (11,7.25);
\draw [short] (7.5,11.75) -- (7.5,7.25);
\draw [short] (8.75,11.75) -- (8.75,7.25);
\draw [short] (10,11.75) -- (10,7.25);
\draw [short] (6.25,9.5) -- (11,9.5);
\draw [short] (6.25,10.75) -- (11,10.75);
\draw [short] (6.25,8.5) -- (11,8.5);
\node [font=\large] at (7,11.25) {1};
\node [font=\LARGE] at (8,11.25) {$\times$};
\node [font=\large] at (9.5,11.25) {4};
\node [font=\large] at (10.5,11.25) {3};
\node [font=\LARGE] at (6.75,10) {$\times$};
\node [font=\Large] at (8,10) {5};
\node [font=\Large] at (9.25,10) {5};
\node [font=\Large] at (10.5,10) {4};
\node [font=\Large] at (7,9) {5};
\node [font=\LARGE] at (8.25,9) {$\times$};
\node [font=\Large] at (9.5,9) {6};
\node [font=\LARGE] at (10.5,9) {$\times$};
\end{circuitikz}
}%

\label{fig:my_label}
\end{figure}
The number in a cell represents the count of the immediate neighboring cells $\brak{\text{left, right, top, bottom, diagonals}}$ NOT having a cross$\brak{\times}$. Given that the last row has no crosses$\brak{\times}$, the sum of the four numbers to be filled in the last row is
\begin{multicols}{4}
\begin{enumerate}
\item $11$
\item $10$
\item $12$
\item $9$
\end{enumerate}
\end{multicols}
}
\item{
Let $D$ be the region bounded by the line $y=x$ and the parabola $y=4x-x^2$. Then $\int\int_Dxdxdy$ equals
\begin{multicols}{4}
\begin{enumerate}
\item $\frac{27}{4}$
\item $\frac{29}{4}$
\item $7$
\item $6$
\end{enumerate}
\end{multicols}
}
\item{
Let $\cbrak{a_n}_{n\geq 1}$ be a sequence of real numbers such that $a_1=\sqrt{6}$ and $a_{n+1}=\sqrt{6+a_n}$ for $n\geq 1$. Consider the following statements:\\
$\brak{I}\cbrak{a_n}_{n\geq 1}$ is an increasing sequence.\\
$\brak{II}\lim_{n\to\infty}a_n=2$\\
Which of the above statements is/are true?
\begin{multicols}{2}
    \begin{enumerate}
        \item Only $\brak{I}$
        \item Only $\brak{II}$
        \item Both $\brak{I}$ and $\brak{II}$
        \item Neither $\brak{I}$ nor $\brak{II}$
    \end{enumerate}
\end{multicols}
}
\item{
Let $A$ be a $3\times 3$ real matrix and let $I_3$ be the $3\times 3$ idendity matrix. Which of the following statements is NOT true?
\begin{enumerate}
    \item If the row-reduced echelon form of $A$ is $I_3$, then zero is not an eigenvalue of $A$
    \item If zero is not an eigenvalue of $A$, then the row-reduced echelon form of $A$ is $I_3$
    \item If $A$ has three distinct eigenvalues, then the row-reduced echelon form of $A$ is $I_3$
    \item If the system of equations $Ax=b$ has a solution for every $3\times 1$ real column vector $b$, then the row-reduced echelon form of $A$ is $I_3$
\end{enumerate}

}
