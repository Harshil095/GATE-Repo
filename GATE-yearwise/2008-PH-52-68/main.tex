\iffalse
	\title{2008-PH-52-68}
	\author{EE24Btech11006 - Arnav Mahishi}
	\section{ph}
	\chapter{2008}
\fi
\item{
When the refractive index $\mu$ of the active medium changes by $\Delta\mu$ in a laser resonator of length $L$, the change in the spectral spacing between the longitudinal modes of the laser is $\brak{\text{ c is the speed of light in free space}}$ 
\begin{multicols}{2}
\begin{enumerate}
\item$\frac{c}{2\brak{\mu+\Delta\mu}L}$
\item$\frac{c}{2\Delta\mu L}$
\item$\frac{c\Delta\mu}{2L\mu\brak{\mu+\Delta\mu}}$
\item zero
\end{enumerate}
\end{multicols}
}
\item{
The primitive translation vectors of the body centered cubic lattice are $\overrightarrow{a}=\frac{a}{2}\brak{\hat{x}-\hat{y}+\hat{z}}$. The primitive translation vectors $\overrightarrow{A},\overrightarrow{B}$, and $\overrightarrow{C}$ of the reciprocal lattice are
\begin{enumerate}
\item $\overrightarrow{A}=\frac{2\pi}{a}\brak{\hat{x}-\hat{y}};\overrightarrow{B}=\frac{2\pi}{a}\brak{\hat{y}+\hat{z}};\overrightarrow{C}=\frac{2\pi}{a}\brak{\hat{x}+\hat{z}}$
\item $\overrightarrow{A}=\frac{2\pi}{a}\brak{\hat{x}+\hat{y}};\overrightarrow{B}=\frac{2\pi}{a}\brak{\hat{y}-\hat{z}};\overrightarrow{C}=\frac{2\pi}{a}\brak{\hat{x}+\hat{z}}$
\item $\overrightarrow{A}=\frac{2\pi}{a}\brak{\hat{x}+\hat{y}};\overrightarrow{B}=\frac{2\pi}{a}\brak{\hat{y}+\hat{z}};\overrightarrow{C}=\frac{2\pi}{a}\brak{\hat{x}-\hat{z}}$
\item $\overrightarrow{A}=\frac{2\pi}{a}\brak{\hat{x}+\hat{y}};\overrightarrow{B}=\frac{2\pi}{a}\brak{\hat{y}+\hat{z}};\overrightarrow{C}=\frac{2\pi}{a}\brak{\hat{x}+\hat{z}}$
\end{enumerate}
\item{
The structure factor of a single cell of identical atoms of form factor $f$ is given by $S_{hkl}=f\sum_jexp\brak{-i2\pi\brak{x_jh+y_jk+z_jl}}$ where $\brak{x_j,y_j,z_j}$ is the coordinate of an atom, and $hkl$ are the Miller indices. Which one of the following statement is correct for the differentiation peaks of the body centered cubic $\brak{\text{BCC}}$ and face centered cubic $\brak{\text{FCC}}$ lattices?
\begin{multicols}{2}
\begin{enumerate}
\item BCC: $\brak{200};\brak{110};\brak{222}$\\FCC: $\brak{111};\brak{311};\brak{400}$
\item BCC: $\brak{210};\brak{110};\brak{222}$\\FCC: $\brak{111};\brak{311};\brak{400}$
\item BCC: $\brak{200};\brak{110};\brak{222}$\\FCC: $\brak{111};\brak{211};\brak{400}$
\item BCC: $\brak{200};\brak{210};\brak{222}$\\FCC: $\brak{111};\brak{211};\brak{400}$
\end{enumerate}
\end{multicols}
}
\item{
The lattice specific heat $C$ of a crystalline solid can be obtained using the Dulong Petit model, Einstie model and Debye model. At low temperature $\hbar\omega\textgreater\textgreater k_bT$, which one of the following statements are true $\brak{\text{a and A are constants}}$
\begin{enumerate}
\item Dulong Petit: $C\propto exp\brak{\frac{-a}{T}}$; Einstein: $C=constant$;Debye: $C\propto\brak{\frac{T}{A}}^3$
\item Dulong Petit: $C=constant$; Einstein:$C\propto\brak{\frac{T}{A}}^3$;Debye:  $C\propto exp\brak{\frac{-a}{T}}$
\item Dulong Petit: $C\propto exp\brak{\frac{-a}{T}}$; Einstein:  $C\propto\frac{e^{\frac{-a}{T}}}{T^2}$;Debye: $C\propto\brak{\frac{T}{A}}^3$
\item Dulong Petit:$C\propto\brak{\frac{T}{A}}^3$; Einstein:$C\propto\frac{e^{\frac{-a}{T}}}{T^2}$;Debye: $C=constant$ 
\end{enumerate}
}
\item{
A linear diatomic lattice of lattice constant a with masses $M$ and $m\brak{M\textgreater m}$ are coupled by a force constant $C$. The dispersion relation is given by\\
$\omega_\pm^2=C\brak{\frac{M+m}{Mm}}\pm\sbrak{C^2\brak{\frac{M+m}{Mm}}^2-\frac{4C^2}{Mm}\sin^2\frac{ka}{2}}^{\frac{1}{2}}$\\
Which of the follwing statements is INCORRECT?
\begin{enumerate}
\item The atoms vibrating in transverse mode correspond to the optical branch.
\item The maximum frequency of the acoustic branch depends on the mass of the lighter atom $m$.
\item The dispersion of frequency in the optical branch is smaller than that in the acoustic branch.
\item No normal modes exist in the acoustic branch for any frequency greater than the maximum frequency at $k=\frac{\pi}{a}$.
\end{enumerate}
}
\item{
The kinetic energy of a free electron at a corner of the first Brillouin zone of a two dimensional square lattice is larger than that of an electron at the mid-point of a side of the zone by a factor $b$. The value of $b$ is 
\begin{multicols}{4}
\begin{enumerate}
\item $b=\sqrt{2}$
\item $b=2$
\item $b=4$
\item $b=8$
\end{enumerate}
\end{multicols}
}
\item{
An intrinsic semiconductor with a mass of a hole $m_h$ and mass of an electron $m_e$ is at a finite temperature $T$. If the top of the valence band energy is $E_v$ and the bottom of the conduction band energy is $E_c$, the Fermi energy of the semiconductor is  
\begin{multicols}{2}
\begin{enumerate}
\item $E_F=\brak{\frac{E_v+E_c}{2}}-\frac{3}{4}k_bT\ln\brak{\frac{m_h}{m_e}}$
\item $E_F=\brak{\frac{k_bT}{2}}+\frac{3}{4}\brak{E_v+E_c}\ln\brak{\frac{m_h}{m_e}}$
\item $E_F=\brak{\frac{E_v+E_c}{2}}+\frac{3}{4}k_bT\ln\brak{\frac{m_h}{m_e}}$
\item $E_F=\brak{\frac{k_bT}{2}}-\frac{3}{4}\brak{E_v+E_c}\ln\brak{\frac{m_h}{m_e}}$
\end{enumerate}
\end{multicols}
}
\item{
Choose the correct statement from the following:
\begin{enumerate}
\item The reaction $K^+K^-\rightarrow p\overline{p}$ can proceed irrespective of the kinetic energies of $K^+$ and $K^-$
\item The reaction $K^+K^-\rightarrow p\overline{p}$ is forbidden by the baryon number conservation
\item The reaction $K^+K^-\rightarrow 2\gamma$ is forbidden by strangeness conservation.
\item The decay $K^0\rightarrow\pi^+\pi^-$ proceeds via weak interactions.
\end{enumerate}
}
\item{
The following gives a list of pairs containing $\brak{i}$ a nucleus $\brak{ii}$ one of its properties. Find the pair which is INAPPROPRIATE.
\begin{enumerate}
\item $\brak{i}$ $_{10}Ne^{20}$ nucleus;$\brak{ii}$ stable nucleus 
\item $\brak{i}$ A spheroidal nucleus;$\brak{ii}$ an electric quadrupole moment
\item $\brak{i}$ $_8O^{16}$ nucleus;$\brak{ii}$ nuclear spin J=$\frac{1}{2}$
\item $\brak{i}$ $U^{238}$ nucleus; $\brak{ii}$ Binding energy=$1785$ MeV $\brak{\text{approximately}}$
\end{enumerate}
}
\item{
The four possible configurations of neutrons in the ground state of $_4Be^9$ nucleus, according to the shell model, and the associated nuclear spin are listed below. Choose the correct one
\begin{multicols}{2}
\begin{enumerate}
\item $\brak{1s_\frac{1}{2}}^2\brak{1p_\frac{3}{2}}^3;J=\frac{3}{2}$
\item $\brak{1s_\frac{1}{2}}^2\brak{1p_\frac{1}{2}}^2\brak{1p_\frac{3}{2}}^1;J=\frac{3}{2}$
\item $\brak{1s_\frac{1}{2}}^1\brak{1p_\frac{3}{2}}^4;J=\frac{1}{2}$
\item $\brak{1s_\frac{1}{2}}^2\brak{1p_\frac{3}{2}}^2\brak{1p_\frac{1}{2}}^1;J=\frac{1}{2}$
\end{enumerate}
\end{multicols}
}
\item{
The mass difference between the pair of mirror nuclei $_6C^11$ and $_5B^{11}$ is given to be $\Delta\frac{MeV}{c^2}$. According to the semi-empirical mass formula, the mass difference between the pair of mirror nuclei $_9F^{17}$ and $_8O^{17}$ will approximately be $\brak{\text{rest mass of proton }m_p=938.27\frac{MeV}{c^2}\text{ and rest mass of neutron }m_n=939.57\frac{MeV}{c^2}}$
\begin{multicols}{2}
\begin{enumerate}
\item $1.39\Delta\frac{MeV}{c^2}$ 
\item $\brak{1.39\Delta+0.5}\frac{MeV}{c^2}$
\item $0.86\Delta\frac{MeV}{c^2}$ 
\item $\brak{1.6\Delta+0.78}\frac{MeV}{c^2}$
\end{enumerate}
\end{multicols}
}
\item{
A heavy nucleus is found to contain more neutrons than protons. This fact is related to which one of the following statements
\begin{enumerate}
\item The nuclear force between neutrons is stronger than that between protons
\item The nuclear force between protons is of a shorter range than those between neutrons, so that a smaller number of protons are held together by the nuclear force.
\item Protons are unstable, so their number in a nucleus diminishes
\item It costs more energy to add a proton to a $\brak{\text{heavy}}$ nucleus than a neutron because of the coulomb repulsion between protons.
\end{enumerate}
}
\item{
A neutral pi meson $\brak{\pi^0}$ has a rest-mass of approximately $140\frac{MeV}{c^2}$ and a lifetime of $\tau$ sec. A $\pi^0$ produced in the laboratory is found to decay after $1.25\tau$ sec into two photons. Which of the follwing sets represents a possible set of energies of the two photons as seen in the laboratory?
\begin{multicols}{2}
\begin{enumerate}
\item $70$ MeV and $70$ MeV
\item $35$ MeV and $100$ MeV
\item $75$ MeV and $100$ MeV  
\item $25$ MeV and $150$ MeV
\end{enumerate}
\end{multicols}
}
\item{
An $a\cdot c$ voltage of $220$ $V_{rms}$ is applied to the primary of a $10:1$ step-down transformer. The secondary of the transformer is centre tapped and connected to a full wave rectifier with a load resistance. The $d\cdot c$ voltage appearing across the load is 
\begin{enumerate}
\item $\frac{22}{\pi}$
\item $\frac{31}{\pi}$
\item $\frac{62}{\pi}$
\item $\frac{44}{\pi}$
\end{enumerate}
}
\item{
Let $I_1$ and $I_2$ represent mesh currents in the loop $abcda$ and $befcb$ respectively. The correct expression describing Kirchoff's voltage loop law in one of the follwing loops is,
\begin{figure}[H]
\centering
\resizebox{5cm}{!}{%
\begin{circuitikz}
\tikzstyle{every node}=[font=\normalsize]
\draw (5.25,8.75) to[short] (5.25,9.25);
\draw (5.25,9.25) to[american current source] (5.25,11.25);
\draw (5.25,8.75) to[short] (6.5,8.75);
\draw (5.25,11.25) to[short] (6.5,11.25);
\draw (6.5,8.75) to[short] (6.5,9.25);
\draw (6.5,11.25) to[short] (6.5,10.75);
\draw (6.5,10.75) to[R] (6.5,9.25);
\draw (6.5,11.25) to[short] (7,11.25);
\draw (7,11.25) to[R] (8,11.25);
\draw (6.5,8.75) to[short] (8.5,8.75);
\draw (8,11.25) to[short] (8.5,11.25);
\draw (8.5,11.25) to[short] (8.5,10.75);
\draw (8.5,8.75) to[short] (8.5,9.25);
\draw (8.5,10.75) to[R] (8.5,9.25);
\draw (8.5,8.75) to[short] (10.5,8.75);
\draw (8.5,11.25) to[short] (9,11.25);
\draw (9,11.25) to[R] (10,11.25);
\draw (10,11.25) to[short] (10.5,11.25);
\draw (10.5,11.25) to[short] (10.5,10.5);
\draw (10.5,10.5) to[american voltage source] (10.5,9.5);
\draw (10.5,9.5) to[short] (10.5,8.75);
\node [font=\normalsize] at (4.5,10.25) {2A};
\node [font=\normalsize] at (6,10) {5$\Omega$};
\node [font=\normalsize] at (7.5,11.75) {10$\Omega$};
\node [font=\normalsize] at (9,10) {15$\Omega$};
\node [font=\normalsize] at (9.5,11.75) {20$\Omega$};
\node [font=\normalsize] at (11.25,10) {20V};
\node [font=\normalsize] at (6.5,11.5) {a};
\node [font=\normalsize] at (6.5,8.5) {d};
\node [font=\normalsize] at (8.5,8.5) {c};
\node [font=\normalsize] at (8.5,11.5) {b};
\node [font=\normalsize] at (10.5,8.5) {f};
\node [font=\normalsize] at (10.5,11.5) {e};
\end{circuitikz}
}%
\label{Circuit}
\end{figure}
\begin{multicols}{2}
\begin{enumerate}
\item $30I_1+5I_2=10$
\item $-15I_1+20I_2=-20$
\item $30I_1-15I_2=10$
\item $-15I_1+20I_2=20$    
\end{enumerate}
\end{multicols}
}
\item{
The simplest logic gate circuit corresponding to the Boolean expression, $Y=P+\overline{P}Q$ is
\begin{multicols}{2}
\begin{enumerate}
    \item{
    \begin{figure}[H]
\centering
\resizebox{3cm}{!}{%
\begin{circuitikz}
\tikzstyle{every node}=[font=\normalsize]

\draw (8,9.25) to[short] (8.25,9.25);
\draw (8,8.75) to[short] (8.25,8.75);
\draw (8.25,9.25) node[ieeestd and port, anchor=in 1, scale=0.89](port){} (port.out) to[short] (10,9);
\draw [short] (8,9.25) -- (7.25,9.25);
\draw [short] (8,8.75) -- (7.25,8.75);
\draw [short] (10,9) -- (10.25,9);
\draw [short] (10.25,9) -- (10.75,9);
\node [font=\normalsize] at (7,9.5) {P};
\node [font=\normalsize] at (7,8.5) {Q};
\node [font=\normalsize] at (11,9) {Y};
\end{circuitikz}
}%

\label{fig:my_label}
\end{figure}
    }
    \item {
    \begin{figure}[H]
\centering
\resizebox{5cm}{!}{%
\begin{circuitikz}
\tikzstyle{every node}=[font=\normalsize]

\draw [short] (4.5,11) -- (4.75,11);
\draw [short] (4.75,11) -- (5.75,11);
\draw [short] (5,11) -- (5,10);
\draw [short] (5,10) -- (5,9.25);
\draw [short] (5,9.25) -- (6.25,9.25);
\draw (6.5,9.25) node[ieeestd not port, anchor=in](port){} (port.out) to[short] (8.25,9.25);
\draw (port.in) to[short] (6.25,9.25);
\draw [short] (5.75,11) -- (6.5,11);
\draw [short] (8.25,9.25) -- (9,9.25);
\draw (9,9.25) to[short] (9.25,9.25);
\draw (9,8.75) to[short] (9.25,8.75);
\draw (9.25,9.25) node[ieeestd or port, anchor=in 1, scale=0.89](port){} (port.out) to[short] (11,9);
\draw [short] (9,8.75) -- (5,8.75);
\draw [short] (5,8.75) -- (4.5,8.75);
\draw [short] (6.5,11) -- (6.5,11.25);
\draw [short] (6.5,11) -- (6.5,10.75);
\draw (6.5,11.25) to[short] (6.75,11.25);
\draw (6.5,10.75) to[short] (6.75,10.75);
\draw (6.75,11.25) node[ieeestd and port, anchor=in 1, scale=0.89](port){} (port.out) to[short] (8.5,11);
\draw [short] (8.5,11) -- (10.75,11);
\draw [short] (10.75,11) -- (11,11);
\draw [short] (11,11) -- (11,10.5);
\draw (11.75,10.25) to[short] (12,10.25);
\draw (11.75,9.75) to[short] (12,9.75);
\draw (12,10.25) node[ieeestd or port, anchor=in 1, scale=0.89](port){} (port.out) to[short] (13.75,10);
\draw [short] (11.75,10.25) -- (11,10.25);
\draw [short] (11,10.5) -- (11,10.25);
\draw [short] (11,9) -- (11.25,9);
\draw [short] (11.25,9) -- (11.25,9.75);
\draw [short] (11.25,9.75) -- (11.75,9.75);
\node [font=\normalsize] at (4.25,11) {P};
\node [font=\normalsize] at (4.25,8.75) {Q};
\node [font=\normalsize] at (14,10) {Y};
\end{circuitikz}
}%

\label{fig:my_label}
\end{figure}
}
    \item{
    \begin{figure}[H]
\centering
\resizebox{3cm}{!}{%
\begin{circuitikz}
\tikzstyle{every node}=[font=\normalsize]

\draw [short] (8,9.25) -- (7.25,9.25);
\draw [short] (8,8.75) -- (7.25,8.75);
\draw [short] (10,9) -- (10.25,9);
\draw [short] (10.25,9) -- (10.75,9);
\node [font=\normalsize] at (7,9.5) {P};
\node [font=\normalsize] at (7,8.5) {Q};
\node [font=\normalsize] at (11,9) {Y};
\draw (8,9.25) to[short] (8.25,9.25);
\draw (8,8.75) to[short] (8.25,8.75);
\draw (8.25,9.25) node[ieeestd or port, anchor=in 1, scale=0.89](port){} (port.out) to[short] (10,9);
\end{circuitikz}
}%

\label{fig:my_label}
\end{figure}
    }
    \item{
    \begin{figure}[H]
\centering
\resizebox{5cm}{!}{%
\begin{circuitikz}
\tikzstyle{every node}=[font=\normalsize]

\draw [short] (4.5,11) -- (4.75,11);
\draw [short] (4.75,11) -- (5.75,11);
\draw [short] (5,11) -- (5,10);
\draw [short] (5,10) -- (5,9.25);
\draw [short] (5,9.25) -- (6.25,9.25);
\draw (6.5,9.25) node[ieeestd not port, anchor=in](port){} (port.out) to[short] (8.25,9.25);
\draw (port.in) to[short] (6.25,9.25);
\draw [short] (5.75,11) -- (6.5,11);
\draw [short] (8.25,9.25) -- (9,9.25);
\draw [short] (9,8.75) -- (5,8.75);
\draw [short] (5,8.75) -- (4.5,8.75);
\draw [short] (6.5,11) -- (6.5,11.25);
\draw [short] (6.5,11) -- (6.5,10.75);
\draw [short] (8.5,11) -- (10.75,11);
\draw [short] (10.75,11) -- (11,11);
\draw [short] (11,11) -- (11,10.5);
\draw [short] (11.75,10.25) -- (11,10.25);
\draw [short] (11,10.5) -- (11,10.25);
\draw [short] (11,9) -- (11.25,9);
\draw [short] (11.25,9) -- (11.25,9.75);
\draw [short] (11.25,9.75) -- (11.75,9.75);
\node [font=\normalsize] at (4.25,11) {P};
\node [font=\normalsize] at (4.25,8.75) {Q};
\node [font=\normalsize] at (14,10) {Y};
\draw (6.5,11.25) to[short] (6.75,11.25);
\draw (6.5,10.75) to[short] (6.75,10.75);
\draw (6.75,11.25) node[ieeestd or port, anchor=in 1, scale=0.89](port){} (port.out) to[short] (8.5,11);
\draw (9,9.25) to[short] (9.25,9.25);
\draw (9,8.75) to[short] (9.25,8.75);
\draw (9.25,9.25) node[ieeestd and port, anchor=in 1, scale=0.89](port){} (port.out) to[short] (11,9);
\draw (11.75,10.25) to[short] (12,10.25);
\draw (11.75,9.75) to[short] (12,9.75);
\draw (12,10.25) node[ieeestd and port, anchor=in 1, scale=0.89](port){} (port.out) to[short] (13.75,10);
\end{circuitikz}
}%

\label{fig:my_label}
\end{figure}
    }
\end{enumerate}
\end{multicols}
}
\item{
An analog voltage $V$ is converted into $2$-bit binary number. The minimum number of comparators required and their reference voltages are
\begin{multicols}{2}
\begin{enumerate}
    \item $3,\brak{\frac{V}{4},\frac{V}{2},\frac{3V}{4}}$
    \item $3,\brak{\frac{V}{3},\frac{2V}{3},V}$
    \item $4,\brak{\frac{V}{5},\frac{2V}{5},\frac{3V}{5},\frac{4V}{5}}$
    \item $4,\brak{\frac{V}{4},\frac{V}{2},\frac{3V}{4},V}$
\end{enumerate}
\end{multicols}
}
}
