\iffalse
	\title{2008-MA-35-51}
	\author{EE24Btech11006 - Arnav Mahishi}
	\section{ma}
	\chapter{2008}
\fi
\item{
Let $y_1$ and $y_2$ be two linearly independent solutions of $y^{\prime\prime}+\brak{\sin x}y,0\leq x\leq1$. Let $g\brak{x}=W\brak{y_1,y_2}\brak{x}$ be the Wronkskian of $y_1$ and $y_2$.Then   
\begin{multicols}{2}
\begin{enumerate}
\item$g^{\prime}\textgreater0$ on $\sbrak{0,1}$
\item$g^{\prime}\textless0$ on $\sbrak{0,1}$
\item$g^{\prime}$ vanishes at only one point of $\brak{0,1}$
\item$g^{\prime}$ vanishes at all points of $\sbrak{0,1}$
\end{enumerate}
\end{multicols}
}
\item{
One particular solution of $y^{\prime\prime\prime}-y^{\prime\prime}-y^{\prime}+y=-e^x$ is constant multiple of 
\begin{multicols}{4}
\begin{enumerate}
\item $xe^{-x}$
\item $xe^x$
\item $x^2e^{-x}$
\item $x^2e^x$
\end{enumerate}
\end{multicols}}
\item{
Let $a,b\in\mathbb{R}$. Let $y=\brak{y_1,y_2}^{\prime}$ be a solution of the system of equations $y_1=y2,y_2^{\prime}=ay_1+by_2$. Every solution $y\brak{x}\rightarrow 0$ as $x\rightarrow\infty$ if
\begin{multicols}{4}
\begin{enumerate}
\item $a\textless0,b\textgreater0$
\item $a\textless0,b\textgreater0$
\item $a\textgreater0,b\textgreater0$
\item $a\textgreater0,b\textless0$
\end{enumerate}
\end{multicols}
}
\item{
Let $G$ be a group of order $45$. Let $H$ be a $3$-Sylow subgroup of $G$ and $K$ be a $5$-Sylow subgroup of $G$. Then
\begin{multicols}{2}
\begin{enumerate}
\item both $H$ and $K$ are normal in $G$ 
\item $H$ is normal in $G$ but $K$ is not normal in $G$
\item $H$ is not normal in $G$ but $K$ is normal in $G$
\item both $H$ and $K$ are not normal in $G$
\end{enumerate}
\end{multicols}
}
\item{
The ring $\mathbb{Z}\sbrak{\sqrt{-11}}$ is 
\begin{multicols}{2}
\begin{enumerate}
\item a Euclidean Domain
\item a Principal Ideal Domain, but not a Euclidean Domain
\item a Unique Factorization Domain, but not a Principal Ideal Domain
\item not a Unique Factorization Domain
\end{enumerate}
\end{multicols}
}
\item{
Let $R$ be a Principal Ideal Domain and $a,b$ any two non-unit elements of $R$. Then the ideal generated by $a$ and $b$ is also generated by
\begin{multicols}{4}
\begin{enumerate}
\item $a+b$
\item $ab$
\item $gcd\brak{a,b}$
\item $lcm\brak{a,b}$
\end{enumerate}
\end{multicols}
}
\item{
Consider the action of $S_4$, the symmetric group of order $4$, on $\mathbb{Z}\sbrak{x_1,x_2,x_3,x_4}$ given by $\sigma\cdotp\brak{x_1,x_2,x_3,x_4}=p\brak{x_{\sigma\brak{1}},x_{\sigma\brak{2}},x_{\sigma\brak{3}},x_{\sigma\brak{4}}}$ for $\sigma\in S_4$. Let $Hsubseteq S_4$ denote the cyclic subgroup generated by $\brak{1,4,2,3}$. Then the cardinality of the orbit $O_H\brak{x_1x_3+x_2x_4}$ of $H$ on the polynomial $x_1x_3+x_2x_4$ is
\begin{multicols}{4}
\begin{enumerate}
\item $1$
\item $2$
\item $3$
\item $4$
\end{enumerate}
\end{multicols}
}
\item{
Let $f:l^2\rightarrow\mathbb{R}$ be defined by $f\brak{x_1,x_2,\cdots}=\sum_{n=1}^\infty\frac{x_n}{2^{\frac{n}{2}}}$ for $\brak{x_1,x_2,\cdots}\in l^2$.Then $\norm{f}$ is equal to
\begin{multicols}{4}
\begin{enumerate}
\item $\frac{1}{2}$
\item $1$
\item $2$
\item $\frac{1}{\sqrt{2}-1}$
\end{enumerate}
\end{multicols}
}
\item{
Consider $\mathbb{R}^3$ with norm $\norm{\text{ }}_1$, and the linear transformation $T:\mathbb{R}^3\rightarrow\mathbb{R}^3$ defined by the $3\times3$ matrix \myvec{1&1&3\\2&2&2\\1&3&-3}. Then the operator norm $\norm{T}$ of $T$ is equal to
\begin{multicols}{4}
\begin{enumerate}
\item $6$
\item $7$
\item $8$
\item $42$
\end{enumerate}
\end{multicols}
}
\item{
Consider $\mathbb{R}^2$ with norm $\norm{\text{ }}_\infty$, and let $Y=\cbrak{\brak{y_1,y_2}\in \mathbb{R}^2:y_1+y_2=0}$. If $g:Y\rightarrow\mathbb{R}$ is defined by $g\brak{y_1,y_2}=y2$ for $\brak{y_1,y_2}\in Y$, then
\begin{enumerate}
\item $g$ has no Hahn-Banach extension to $\mathbb{R}^2$
\item $g$ has a unique Hahn-Banach extension to $\mathbb{R}^2$
\item every linear functional $f:\mathbb{R}^2\rightarrow\mathbb{R}$ satisfying $f\brak{-1,1}=1$ is a Hahn-Banach extension of $g$ to $\mathbb{R}^2$ 
\item the functionals $f_1,f_2:\mathbb{R}^2\rightarrow\mathbb{R}$ given by $f_1\brak{x_1,x_2}=x_2$ and $f_2\brak{x_1,x_2}=-x_1$ are both Hahn-Banach extensions of $g$ to $\mathbb{R}^2$ 
\end{enumerate}
}
\item{
Let $X$ be a Banach space and $Y$ be a normed linear space. Consider a sequence $\brak{F_n}$ of bounded linear maps from $X$ to $Y$ such that for each fixed $x\in X$, the sequence $\brak{F_n\brak{x}}$ is bounded in $Y$.Then
\begin{enumerate}
\item for each fixed $x\in X$, the sequence $\brak{F_n\brak{x}}$ is convergent in $Y$
\item for each fixed $n\in N$, the set $\cbrak{F_n\brak{x}:x\in X}$ is bounded in $Y$
\item the sequence $\brak{\norm{F_n}}$ is bounded in $\mathbb{R}$
\item the sequence $\brak{F_n}$ is uniformly bounded on $X$
\end{enumerate}
}
\item{
Let $H=L^2\brak{\sbrak{0,\pi}}$ with the usual inner product. For $n\in \mathbb{N}$, let $u_n\brak{t}=\frac{\sqrt{2}}{\sqrt{\pi}}\sin nt,t\in\sbrak{0,\pi}$, and $E=\cbrak{u_n:n\in\mathbb{N}}$. Then
\begin{enumerate}
\item $E$ is not a linearly dependent subset of $H$
\item $E$ is a linearly independent subset of $H$, but is not a orthonormal subset of $H$.
\item $E$ is a orthonormal subset of $H$, but is not an orthonormal basis for $H$.
\item $E$ is an orthonormal basis for $H$.
\end{enumerate}
}
\item{
Let $X=\mathbb{R}$ and let $\tau=\cbrak{U\subseteq X:X-U\text{ is finite}}\cup\cbrak{\phi,X}$. The sequence $1,\frac{1}{2},\frac{1}{3},\cdots,\frac{1}{n},\cdots$ in $\brak{X,\tau}$
\begin{multicols}{2}
\begin{enumerate}
\item converges to $0$ and not to any other point of $X$
\item does not converge to $0$
\item converges to each point of $X$  
\item is not convergent to $X$
\end{enumerate}
\end{multicols}
}
\item{
Let $E=\cbrak{\brak{x,y}\in\mathbb{R}^2:\abs{x}\leq 1,\abs{y}\leq 1}$. Define $f:E\rightarrow\mathbb{R}$ by $f\brak{x,y}=\frac{x+y}{1+x^2+y^2}$. Then the range of $f$ is a
\begin{enumerate}
\item connected open set
\item connected closed set
\item bounded open set
\item closed and unbounded set
\end{enumerate}
}
\item{
Let $X=\cbrak{1,2,3}$ and $\tau=\cbrak{\phi,\cbrak{1},\cbrak{2},\cbrak{1,2},\cbrak{2,3},\cbrak{1,2,3}}$. The topological space $\brak{X,\tau}$ is said to have the property $P$ if for any two proper disjoint closed subsets $Y$ and $Z$ of $X$, there exist disjoint oper sets $U,V$ such that $Y\subseteq U$ and $Z\subseteq V$. Then the topological space $\brak{X,\tau}$
\begin{multicols}{2}
\begin{enumerate}
\item is $T_1$ and satisfies $P$
\item is $T_1$ and does not satisfy $P$
\item is not $T_1$ and satisfies $P$
\item is $T_1$ and does not $P$
\end{enumerate}
\end{multicols}
}
\item{
Which one of the follwing subsets of $\mathbb{R}$ $\brak{\text{with the usual metric}}$ is NOT complete?
\begin{multicols}{4}
\begin{enumerate}
    \item $\sbrak{1,2}\cup\sbrak{3,4}$
    \item $\sbrak{0,\infty}$
    \item $[0,1)$
    \item $\cbrak{0}\cup\cbrak{\frac{1}{n}:n\in\mathbb{N}}$
\end{enumerate}
\end{multicols}
}
\item{
Consider the function\\
$f\brak{x}=$
$\begin{cases}
k\brak{x-\sbrak{x}}, 0\leq x\textless 2\\
0, \text{otherwise}
\end{cases}$
where $\cbrak{x}$ is the integral part of $x$. The value of $k$ for which the above function is a probability density function of some random variable is
\begin{multicols}{4}
\begin{enumerate}
    \item $\frac{1}{4}$
    \item $\frac{1}{2}$
    \item $1$
    \item $2$
\end{enumerate}
\end{multicols}
}
